\documentclass[acus]{article}



\usepackage{booktabs} 
\usepackage{longtable}
\usepackage{subfig}

\usepackage{graphicx}

\begin{document}

\title{AT 2.0 Manual}
\maketitle
\begin{abstract}
We give an overview for the AT 2.0 code and describe how the functions and repository fit into this.
\end{abstract}


\section{Introduction}
AT is a toolbox of functions in Matlab for charged particle beam simulation.
On the one hand, as a toolbox, one should be allowed to use the tools to do what one wants.
On the other, in order to avoid redundancy in work, and facilitate working together via sharing of tools, it is useful to implement some standards.  We describe the tools and some standards and guidelines in AT with the hope that the overall package has some coherence and elegance (speed is also considered at some points where it is critical).

\section{Lattice Creation}
The element creation functions are the following:
\begin{itemize}
\item atdrift \ \ Class: Drift
\item atmonitor \ \ Class: Monitor
\item atmultipole \ \ Class: Multipole
\item atthinmultipole \ \ Class: 
\item atquadrupole  \ \ Class: Quadrupole
\item atrbend  \ \ Class: Bend 
\item atrfcavity \ \ Class: RFCavity
\item atsbend \ \ Class: Bend
\item at solenoid \ \ Class: Solenoid
\item atsextupole  \ \ Class: Sextupole
\item atwiggler  \ \ Class: Wiggler
\item idtable  \ \ Class:  KickMap
\end{itemize}


\section{Pass Methods}

\begin{itemize}
\item BndMPoleSymplectic4E2Pass
\item BndMPoleSymplectic4E2RadPass
\item BndMPoleSymplectic4Pass
\item BndMPoleSymplectic4RadPass
\item BendLinearPass
\item CavityPass
\item CorrectorPass
\item DriftPass
\item QuadLinearPass
\item QuadMPoleFringePass
\item StrMPoleSymplectic4Pass
\item StrMPoleSymplectic4RadPass
\item ThinMPolePass
\item WiggLinearPass
\item IDTablePass
\end{itemize}



\section{Lattice Manipulation}
A lattice manipulation function takes a lattice as an argument and produces a new lattice as a result.
Here are some lattice manipulation functions:
\begin{itemize}
\item atsetshift
\item atsettilt
\item atsetfieldvalues
\item ataddmpolecomppoly
\item ataddmpoleerrors
\item atloadfielderrs
\end{itemize}


\section{Tracking Particles plus Moments}
The pass methods have two different calling methods.  They may be called directly via the Mex interface (through the MexFunction entry point in the C function), or they may be called indirectly through the function RingPass (through the passFunction entry).  The pass methods should be defined so that calling them with no arguments gives a list of required and optional parameters.

The moment tracking and equilibrium finding occurs via the function OhmiEnvelope().  For this to work requires pass methods that include radiation (this gives a deterministic effect which results in damping and non-symplecticity.  Further, the function findmpoleraddiffmatrix is required to compute the diffusion matrix.

\section{Global parameters and Lattice Functions}
Given the ability to track particles through the lattice, one can compute global beam dynamics parameters and properties that vary around the ring.  
Here are some global parameters:
\begin{itemize}
\item tunes
\item chromaticity
\item momentum compaction factor
\item tune shift with amplitude
\end{itemize}

Here are some quantities that vary around the ring:
\begin{itemize}
\item Closed orbit
\item One turn map matrix
\item Transfer matrix from one position to another
\item Beam sizes
\item Twiss Parameters
\item Dispersion function
\end{itemize}


\section{Visualization}
The lattice functions described in the previous section may be plotted, together with a synoptic representation of the lattice.  The function atplot is designed for this purpose.

\section{AT within a larger context: Other Codes, Matlab Middle Layer}

\begin{thebibliography}{2}

\bibitem{KMW}
A. Terebilo \emph{Accelerator Toolbox for Matlab}, SLAC-PUB 8732 (May 2001)

\end{thebibliography}

\end{document}

