\documentclass[acus]{article}



\usepackage{booktabs} 
\usepackage{longtable}
\usepackage{subfig}

\usepackage{graphicx}

\begin{document}

\title{AT Manual}
\maketitle
\begin{abstract}
We give an overview for the AT code and describe how the functions and repository fit into this.
\end{abstract}


\section{Introduction}

\section{Lattice Creation}
The element creation functions are the following:
\begin{enumerate}
\item atmultipole
\item atquadrupole
\item atrbend
\item atsbend
\item atsextupole
\item atwiggler
\item idtable
\end{enumerate}



\section{Pass Methods}

\begin{enumerate}
\item BndMPoleSymplectic4E2Pass
\item BndMPoleSymplectic4E2RadPass
\item BndMPoleSymplectic4Pass
\item BndMPoleSymplectic4RadPass
\item BendLinearPass
\item CavityPass
\item CorrectorPass
\item DriftPass
\item QuadLinearPass
\item QuadMPoleFringePass
\item StrMPoleSymplectic4Pass
\item StrMPoleSymplectic4RadPass
\item ThinMPolePass
\item WiggLinearPass
\item IDTablePass
\end{enumerate}



\section{Lattice Manipulation}

\section{Tracking Particles plus Moments}

\section{Lattice Functions}

\section{Visualization}

\section{AT within a larger context: Other Codes, Matlab Middle Layer}

\begin{thebibliography}{2}

\bibitem{KMW}
A. Terebilo \emph{Accelerator Toolbox for Matlab}, SLAC-PUB 8732 (May 2001)

\end{thebibliography}

\end{document}

